% !TEX root = Entwurf_goApp.tex

\section{Server}
	\subsection{edu.kit.sdqweb.pse.gruppe1.goApp.server.servlet}
	
	\subsection{edu.kit.sdqweb.pse.gruppe1.goApp.server.database.model}

	\subsubsection{public class User}
	An user describes an user of the goApp.
	\newline Attributes:
	\begin{itemize}
	\item private int Id : The Id is used to identify each user and is therefore unique.
	\item private String name : The name of an user is selectable by the user and can also be changed.
	\end{itemize}
	Methods: 
	\begin{itemize}
	\item public User(int id, String name)
		\begin{description}
		\item @param id The Id of the user. 
	 	\item @param name The name of the user.
		\end{description}
	\item public void setName(String name)
		\begin{description}
		\item Sets the name of the user
	 	\item @param name The new name of the user.
		\end{description}
	\item public String getName()
		\begin{description}
		\item Returns the name of the user.
	 	\item @return The name of the user.
		\end{description}
	\item public int getId()
		\begin{description}
		\item  Returns the Id of the user.
	 	\item @return The Id of the user.
		\end{description}
	\end{itemize}

	\subsubsection{public class Request}
	A request is created whenever a user wants to join a new group and deleted as soon as the founder adds or rejects the user.
	\newline
	Methods: 
	\begin{itemize}
	\item public Request(User user, Group group)
		\begin{description}
		\item @param user The user who creates the request.
		\item @param group The group which the user wants to join.
		\end{description}
	\end{itemize}
	
	\subsubsection{public class Group}
	A group is a composition of several users of the goApp, in which the users can create events.
	\newline Attributes:
	\begin{itemize}
	\item private int Id : The Id is used to identify each group and is therefore unique.
	\item private String name : The name of a group is given by the founder and can be changed.
	\end{itemize}
	Methods:
	\begin{itemize}
	\item public Group(int id, String name)
		\begin{description}
		\item @param id The Id of the group.
		\item @param name The name of the group.
		\end{description}
	\item public void setName(String name)
		\begin{description}
		\item Sets the name of the group.
	 	\item @param name The new name of the group.
		\end{description}
	\item public String getName()
		\begin{description}
		\item Returns the name of the group.
	 	\item @return The name of the group.
		\end{description}
	\item public int getId()
		\begin{description}
		\item Returns the Id of the Group.
	 	\item @return The Id of the Group.
		\end{description}
	\end{itemize}

	\subsubsection{public class Event}
	An event is created by a user within a specific group.
	\newline Attributes:
	\begin{itemize}
	\item private  int Id : The Id is used to identify each event and is therefore unique.
	\item private String name : The name of an event is given by the creator.
	\item private Time time : The time of an event tells when the event is starting and set by the creator of the event.
	\end{itemize}
	Methods:
	\begin{itemize}
	\item public Event(int id, String name, Time time)
		\begin{description}
		\item @param id The Id of the event.
		\item @param name The name of the event.
		\item @param time The time of the event.
		\end{description}
	\item public synchronized void setTime(Time time)
		\begin{description}
		\item Sets the time of the event.
	 	\item @param time The new time of the event.
		\end{description}
	\item public void setName(String name)
		\begin{description}
		\item Sets the name of the event.
	 	\item @param name The new name of the event.
		\end{description}
	\item public Time getTime()
		\begin{description}
		\item Returns the time of the event.
	 	\item @return The time of the event.
		\end{description}
	\item public String getName()
		\begin{description}
		\item Returns the name of the event.
	 	\item @return The name of the event.
		\end{description}
	\item public int getId()
		\begin{description}
		\item Returns the Id of the event
	 	\item @return The Id of the event.
		\end{description}
	\end{itemize}

	\subsubsection{public class Location}
	An event is created by a user within a specific group.
	\newline Attributes:
	\begin{itemize}
	\item private  int longitude : The longitude value of the position.
	\item private  int latitude : The latitude value of the position.
	\item private String name : The name of a location helps the user to identify if the location is a center point of the group or the original meeting point.
	\end{itemize}
	Methods:
	\begin{itemize}
	\item public Location(int longitude, int latitude, String name)
		\begin{description}
		\item @param longitude The longitude value of the position.
		\item @param latitude The latitude value of the position.
		\item @param name The name of the location.
		\end{description}
	\item public void setName(String name)
		\begin{description}
		\item Sets the name of a location.
	 	\item @param name The new name of the location.
		\end{description}
	\item public void setLatitude(int latitude)
		\begin{description}
		\item Sets the latitude of the location.
	 	\item @param latitude The new latitude value.
		\end{description}
	\item public void setLongitude(int longitude)
		\begin{description}
		\item Sets the longitude value of the location.
	 	\item @param longitude The new longitude value.
		\end{description}
	\item public String getName()
		\begin{description}
		\item Returns the name of the location.
	 	\item @return The name of the location.
		\end{description}
	\item public int getLatitude()
		\begin{description}
		\item Returns the latitude value of the location.
	 	\item @return The latitude value of the location.
		\end{description}
	\item public int getLongitude()
		\begin{description}
		\item Returns the longitude value of the location.
	 	\item @return The longitude value of the location.
		\end{description}
	\end{itemize}

	\subsection{edu.kit.sdqweb.pse.gruppe1.goApp.server.database.management}	
	%TODO: Kurzer Beschreibungstext
	
	%TODO: Hinterher Löschen - Vorlage
	%	\newline Attributes:	
	\begin{itemize}
		\item 
		\begin{description}
			\item 
		\end{description}
	\end{itemize}
	%\newline Methods:
	\begin{itemize}
		\item 
		\begin{description}
			\item 
		\end{description}
	\end{itemize}
	%TODO bis hier
	\subsubsection{DatabaseInitializer}
	Factory for creating Session Factory for DB communication.
	\newline Methods:
	\begin{itemize}
		\item public SessionFactory getFactory()
		\begin{description}
			\item Factory Method
			\item @return SessionFactory Object from Hibernate Library
		\end{description}
	\end{itemize}
	
	\subsubsection{Interface Management}
	Interface to use from all DB Management Classes
	\newline Methods:
	\begin{itemize} 
		\item public boolean delete(int ID)			
		\begin{description}
			\item Delete an entry with ID in DB
			\item @param ID ID from Entry to delete
			\item @return true, if deletion was successfull, otherwise false
		\end{description}
		
		\item void Management(DatabaseInitializer dbInitializer);
		\begin{description}
			\item Constructor which initializes a DB Session
			\item @param dbInitializer DataBaseInitializier Object - Created from Factory
		\end{description}
	\end{itemize}
	\subsubsection{GroupManagement}
	\subsubsection{UserManagement}
	\subsubsection{EventManagement}
	\subsubsection{EventUserManagement}
	\subsubsection{GroupUserManagement}
	\subsubsection{RequestManagement}
  
  
  	\subsection{edu.kit.sdqweb.pse.gruppe1.goApp.server.algorithm}
	
	\subsubsection{Cluster}
	
	%TODO: Kurzer Beschreibungstext
	
	%TODO: Hinterher Löschen - Vorlage
	%	\newline Attributes:	
	\begin{itemize}
		\item 
		\begin{description}
			\item 
		\end{description}
	\end{itemize}
	%\newline Methods:
	\begin{itemize}
		\item 
		\begin{description}
			\item 
		\end{description}
	\end{itemize}
	%TODO bis hier
	
	
	
	\subsubsection{CusterFacade}
	
	%TODO: Kurzer Beschreibungstext
	
	%TODO: Hinterher Löschen - Vorlage
	%	\newline Attributes:	
	\begin{itemize}
		\item 
		\begin{description}
			\item 
		\end{description}
	\end{itemize}
	%\newline Methods:
	\begin{itemize}
		\item 
		\begin{description}
			\item 
		\end{description}
	\end{itemize}
	%TODO bis hier
	
	
	
	
	\subsubsection{abstract Clusterer}
	
	%TODO: Kurzer Beschreibungstext
	
	%TODO: Hinterher Löschen - Vorlage
	%	\newline Attributes:	
	\begin{itemize}
		\item 
		\begin{description}
			\item 
		\end{description}
	\end{itemize}
	%\newline Methods:
	\begin{itemize}
		\item 
		\begin{description}
			\item 
		\end{description}
	\end{itemize}
	%TODO bis hier
	
	
	
	\subsubsection{abstract MidPointAlgo}
	
	%TODO: Kurzer Beschreibungstext
	
	%TODO: Hinterher Löschen - Vorlage
	%	\newline Attributes:	
	\begin{itemize}
		\item 
		\begin{description}
			\item 
		\end{description}
	\end{itemize}
	%\newline Methods:
	\begin{itemize}
		\item 
		\begin{description}
			\item 
		\end{description}
	\end{itemize}
	%TODO bis hier
	
	
	
	\subsubsection{SimpleCentral implements MidPointAlgo}
	
	%TODO: Hinterher Löschen - Vorlage
	%	\newline Attributes:	
	\begin{itemize}
		\item 
		\begin{description}
			\item 
		\end{description}
	\end{itemize}
	%\newline Methods:
	\begin{itemize}
		\item 
		\begin{description}
			\item 
		\end{description}
	\end{itemize}
	%TODO bis hier
	
	
	
	
	\subsubsection{ImportantMidPointCentral implements MidPointAlgo}
	
	%TODO: Kurzer Beschreibungstext
	
	%TODO: Hinterher Löschen - Vorlage
	%	\newline Attributes:	
	\begin{itemize}
		\item 
		\begin{description}
			\item 
		\end{description}
	\end{itemize}
	%\newline Methods:
	\begin{itemize}
		\item 
		\begin{description}
			\item 
		\end{description}
	\end{itemize}
	%TODO bis hier
	
	\newpage