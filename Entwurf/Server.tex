% !TEX root = Entwurf_goApp.tex

\section{Server}
	
	\hypertarget{database.model}{}
	\hypertarget{ServerModel}{}
	\subsection{edu.kit.pse.gruppe1.goApp.server.database.model}
Nachfolgend werden die Klassen des Modells mitsamt ihrer Attribute und Methoden detailliert vorgestellt.
	\subsubsection{public class User}
	An user describes an user of the goApp.
	
	Attributes:
	\begin{itemize}
	\item private int id : The id is used to identify each user and is therefore unique and readonly.
	\item private String name : The name of an user is selectable by the user and can also be changed.
	\item private List<Group> groups : All groups the user is a member of.
	\item private Location location : The current location of the user.
	\item private List<Event> events : all events the user has not rejected.
	\end{itemize}
	Methods: 
	\begin{itemize}
	\item public User(int id, String name)
		\begin{description}
		\item @param id The Id of the user. 
	 	\item @param name The name of the user.
		\end{description}
	\item public getters of all attributes
	\item public setters of name and location
	\item public methods for adding elements to the group and event list
	\item public methods for deleting elements from the group and event list
	\end{itemize}


	\subsubsection{public class Request}
	A request is created whenever an user wants to join a new group and deleted as soon as the founder adds or rejects the user.
	
	Attributes:
	\begin{itemize}
	\item private Group group : The group the user wants to join. This attribute is readonly because the request can't be modified.
	\item private User user : The user who wants to join the group. This attribute is readonly because the request can't be modified.
	\end{itemize}
	
	Methods: 
	\begin{itemize}
	\item public Request(User user, Group group)
		\begin{description}
		\item @param user The user who creates the request.
		\item @param group The group which the user wants to join.
		\end{description}
	\item public getters of all attributes
	\end{itemize}
	


\subsubsection {public class Participant} 
 The Participant class describes the status of a participant during the event.

Attributes:
\begin{itemize}
\item	private int status : can be: not participate, not seen, participate, started
\end{itemize}

	Methods: 
	\begin{itemize}
	\item getter and setter of status
	\end{itemize}

	
	\subsubsection{public class Group}
	A group is a composition of several users of the goApp, in which the users can create events.

Attributes:
	\begin{itemize}
	\item private int id : The id is used to identify each group and is therefore unique and readonly.
	\item private String name : The name of a group is given by the founder and can be changed.
	\item private User founder : The founder created the group and can change the name and members. He can also delete the group.
	\item private List<User> members : all users in the group
	\item private List<Event> events : all events in the group
	\item private List<Request> requests  : all requests of the group
	
	\end{itemize}
	Methods:
	\begin{itemize}
	\item public Group(int id, String name)
		\begin{description}
		\item @param id The id of the group.
		\item @param name The name of the group.
		\end{description}
	\item public getters of all attributes
	\item public setters of name and founder
	\item public methods for adding elements to the member, event and request list
	\item public methods for deleting elements from the member, event and request list
	\end{itemize}

	\subsubsection{public class Event}
	An event is created by a user within a specific group.
	\newline Attributes:
	\begin{itemize}
	\item private  int id : The id is used to identify each event and is therefore unique and readonly.
	\item private String name : The name of an event is given by the creator.
	\item private Time time : The time of an event indicates when the event is starting and is set by the creator of the event.
	\item private Group group : The group in which the event has been created. The attribute is readonly.
	\item private Location destination : destination of the event
	\item private User admin : The creator of the event. The admin participates automatically. This attribute is readonly.
	\item private List<User> participants : All members of the group who haven't rejected the event yet.
	\item private List<Location> groupLocations : The locations which were calculated by the clustering algorithm.
	\end{itemize}
	Methods:
	\begin{itemize}
	\item public Event(int id, String name, Time time)
		\begin{description}
		\item @param id The id of the event.
		\item @param name The name of the event.
		\item @param time The time of the event.
		\end{description}
	\item public getters of all attributes
	\item public setters of name, time and destination
	\item public methods for adding elements to the participant and groupLocation list.
	\item public methods for deleting elements from the participant and groupLocation list.
	\end{itemize}

	\subsubsection{public class Location}
	The location is used to display something on the Map, whatever it is, an user or the center of the group.
	\newline Attributes:
	\begin{itemize}
	\item private  double longitude : The longitude value of the position in degrees, range[-180,180). 
	\item private  double latitude : The latitude value of the position in degrees, range [-90,90].
	\item private String name : The name of a location helps the user to identify if the location is a central point of the group or the original meeting point.
	\end{itemize}
	Methods:
	\begin{itemize}
	\item public Location(int longitude, int latitude, String name)
		\begin{description}
		\item @param longitude The longitude value of the position.
		\item @param latitude The latitude value of the position.
		\item @param name The name of the location.
		\end{description}
	\item getters of all attributes
	\item setters of all attributes
	\end{itemize}


\hypertarget{database.management}{}
	\subsection{edu.kit.pse.gruppe1.goApp.server.database.management}	
	Dies sind die Verwaltungsklasse der Datenbank. Nur über diese wird der Zugriff realisiert.

	\subsubsection{DatabaseInitializer}
	Factory for creating Session Factory for DB (=Database) communication.
	\newline 
	Attributes:
	\begin{itemize}
		\item private static SessionFactory factory : This is the only SessionFactory on the Server.
	\end{itemize}
	Methods:
	\begin{itemize}
		\item public static SessionFactory getFactory()
		\begin{description}
			\item Method returns the attribute factory and initialize factory if it is null.
			\item @return SessionFactory object which can create sessions to change the DB.
		\end{description}
	\end{itemize}
	
	\subsubsection{Interface Management}
	Interface to use from all DB Management Classes
	\newline Methods:
	\begin{itemize} 
		\item public boolean delete(int ID)			
		\begin{description}
			\item Deletes an entry with the given ID in the DB
			\item @param ID ID from the entry to delete
			\item @return true, if deletion was successfull and otherwise false
		\end{description}
		\end{itemize}
	
	\subsubsection{GroupManagement}
This class manages the access to the group table.
		\newline Methods:
		\begin{itemize}
			\item public Group add(String name, User founder)
			\begin{description}
				\item creates an new group and adds an new entry to table
				\item  @param name Name of the group
				\item  @param founder Founder of the Group
				\item  @return created Group
			\end{description}
			\item public boolean update(Group chGroup)
			\begin{description}
				\item updates the entry in the table
				\item  @param chGroup Group with changed attributes
				\item  @return true, if update was successfull, otherwise false
			\end{description}
			\item public boolean updateName(int groupID, String newName)
			\begin{description}
				\item updates the name of the entry
				\item @param groupID ID of the entry to change
				\item @param newName Name to set in the DB
				\item @return true, if update was successfull, otherwise false
			\end{description}
			\item public boolean updateFounder(int groupID, User newFounder)
			\begin{description}
				\item updates founder of the given entry
				\item  @param groupID ID of the entry to change
				\item  @param newFounder Founder to set in the DB
			\end{description}
			\item public Group getGroup(int groupID)
			\begin{description}
				\item returns Group with the given GroupID
				\item  @param groupID ID of the group which should be returned
				\item  @return a Group
			\end{description}
			\item public List<Group> getGroupsByMember(String member)
			\begin{description}
				\item get all Groups with the given Member
				\item @param member Name of the Member
				\item @return List of the Groups
			\end{description}
			\item public List<Group> getGroupsByName(String searchName)
			\begin{description}
				\item get all Groups with the given Name
				\item @param searchName name to search for
				\item @return List of the matching Groups
			\end{description}
			\item public boolean addMember(int groupID, int memberId)
			\begin{description}
				\item add Member to the Group
				\item @param groupID ID of the Group to which the new member should be added
				\item @param memberId ID of the User to add to the Group
				\item @return true, if adding was successfull, otherwise false
			\end{description}
			\item public boolean deleteMember(int groupID, int memberID)
			\begin{description}
				\item delete member from the Group
				\item @param groupID ID of Group to delete the member from
				\item @param memberID ID of the member to delete
				\item @return true, if deletion was successfull, otherwise false
			\end{description}
			\item public List<Event> getEvents(int groupID)
			\begin{description}
				\item return all Events which are connected with the given Group
				\item  @param groupID GroupID to search Events for
				\item  @return List of the Events
			\end{description}
		\end{itemize}
		
	\subsubsection{UserManagement}
	This class manages the access to the user table.
		\newline Methods:
		\begin{itemize}
			\item public User add(String name, int googleId)
			\begin{description}
				\item creates a new User and adds a new entry to the table
				\item  @param name name of the User
				\item  @param googleId googleID of the User
				\item  @return userID of the entry
			\end{description}
			\item public boolean update(User chUser)
			\begin{description}
				\item changes the entry in the table
				\item  @param chUser User object with changes
				\item  @return true, if change was successfull, otherwise false
			\end{description}
			\item public boolean updateName(int userID, String newName)
			\begin{description}
				\item updates the Name of the User
				\item @param userID ID from the User to update
				\item @param newName name to set in the DB
				\item @return true, if update was successfull, otherwise false
			\end{description}
			\item public boolean updateLocation(int userId, Location newLocation)
			\begin{description}
				\item updates current location of the User
				\item @param userId userID from the entry to update
				\item @param newLocation new Location of the User
				\item @return true, if update was successfull, otherwise false
			\end{description}
			\item public User getUser(int userID)
			\begin{description}
				\item get User with the given userID
				\item  @param userID ID of the User to search for
				\item  @return found User
			\end{description}
		\end{itemize}
		
	\subsubsection{EventManagement}
This class manages the access to the Event table.
		\newline Methods:
		\begin{itemize}
			\item public Event add(String name, Location location, DateTime time, int userId, int groupID)
			\begin{description}
				\item creates new Group and adds new entry to the table
				\item adds all Users from the Group to the Event and sets the status
				\item @param name Name of the Event
				\item @param location Location, where the Event takes place
				\item @param time Date and Time, when the Event takes place
				\item @param userId ID of the User who created the Event
				\item @param groupID ID of the Group to which the Event is related to
				\item @return created Event
			\end{description}
			\item public boolean update(Event chEvent)
			\begin{description}
				\item updates the entry in the table
				\item  @param chEvent Event with the changes
				\item  @return true, if update was successful, otherwise false
			\end{description}
			\item public boolean updateName(int eventID, String name)
			\begin{description}
				\item updates entry with the given id - sets the new name
				\item  @param eventID ID of the entry to be updated
				\item  @param name new Name of the entry
				\item  @return true, if update was successfull, otherwise false
			\end{description}
			\item public boolean updateStatus(int userID, int eventID, boolean newStatus)
			\begin{description}
				\item sets the new status to the entry with the given ID
				\item  @param userID ID from the User
				\item  @param eventID ID from the Event
				\item  @param newStatus status to set in the DB
				\item  @return true, if update was successfull, otherwise false
			\end{description}
			\item public Event getEvent(int eventID)
			\begin{description}
				\item gets the Event with the given eventID
				\item  @param eventID ID of the Event
				\item  @return matching  Event
			\end{description}
			\item public int getUser(int eventID)
			\begin{description}
				\item get User who created the Event
				\item @param eventID ID of the entry
				\item @return ID of the User
			\end{description}
		\end{itemize}
		
	\subsubsection{EventUserManagement}
This class manages the access to the Event-User table.
		\newline Methods:
		\begin{itemize}
			\item public boolean add(int eventId, int userId)
			\begin{description}
				\item adds new entry to the table
				\item  @param eventId eventID to combine with the User
				\item  @param userId userID to combine with the Event
				\item  @return true, if adding was successfull, otherwise false
			\end{description}
			\item public boolean updateStatus(int eventID, int userID, int newStatus)
			\begin{description}
				\item updates status of the given User and the eventID
				\item  @param eventID eventID to upate in the DB
				\item @param userID userID to update in the DB
				\item  @param newStatus new status to set in the DB
				\item  @return true, if update was successfull, otherwise false
			\end{description}
			\item public boolean delete(int eventID, int userID)
			\begin{description}
				\item removes the entry with the userID and the eventID
				\item  @param eventID eventID from the entry that should be removed.
				\item  @param userID userID from the entry that should be removed.
				\item  @return true, if update was successfull, otherwise false
			\end{description}
			\item public List<Event> getEvents(int userID)
			\begin{description}
				\item get all Events in which the given User participates
				\item @param userID ID of the User
				\item @return List of matching Events
			\end{description}
			\item public List<User> getUsers(int eventID)
			\begin{description}
				\item get all Users who participate in the given Event
				\item  @param eventID ID of the Event
				\item  @return List of matching User
			\end{description}
			\item public List<User> getUserByStatus(String status, int eventID)
			\begin{description}
				\item get all Users with given Status from an Event
				\item @param status status to search for
				\item @param eventID ID of event to search for
				\item @return List of matching Users
			\end{description}
		\end{itemize}
		
	\subsubsection{GroupUserManagement}
	This class manages the access to the Group-User table.
		\newline Methods:
		\begin{itemize}
			\item public boolean add(int gruopID, int userID)
			\begin{description}
				\item creates new entry
				\item  @param gruopID groupID to combine with user
				\item  @param userID userID to combine with group
				\item  @return ID of new entry
			\end{description}
			\item public List<User> getUsers(int gruopID)
			\begin{description}
				\item get all Users to given Group
				\item  @param groupID ID of group
				\item @return List of matching User
			\end{description}
			\item public List<Group> getGroups(int userID)
			\begin{description}
				\item get all groups to given User
				\item @param userID ID of User
				\item @return List of matching Groups
			\end{description}
			\item public boolean delete(int groupID, int userID)
			\begin{description}
				\item deletes an entry in the table with the given groupID and userID
				\item @param groupID groupID to delete (in combination)
				\item @param userID userID to delete (in combination)
				\item @return true, if successful, otherwise false
			\end{description}
		\end{itemize}
		
	\subsubsection{RequestManagement}
This class manages the access to the Request table.
		\newline Methods:
		\begin{itemize}
			\item public boolean add(int gruopID, int userID)
			\begin{description}
				\item creates new entry
				\item @param gruopID groupID to combine with user
				\item @param userID userID to combine with group
				\item @return ID of new entry
			\end{description}
			\item public List<User> getRequestByGroup(int gruopID)
			\begin{description}
				\item get all User to given Gruop
				\item @param gruopID ID of group
				\item @return List of matching Users
			\end{description}
			\item public List<Group> getRequestByUser(int userID)
			\begin{description}
				\item get all groups to given User
				\item @param userID ID of User
				\item @return List of matching Groups
			\end{description}
			\item public boolean delete(int groupID, int userID)
			\begin{description}
				\item delete entry with given groupID and userID
				\item @param groupID
				\item @param userID
				\item @return true, if deletion was successful, otherwise false
			\end{description}
		\end{itemize}
  
  \hypertarget{algorithm}{}
  	\subsection{edu.kit.pse.gruppe1.goApp.server.algorithm}
	
	
	\subsubsection{ClusterFacade}
	
	
	Facade Class which communicates with the servlet and the database.
	
	 Methods:
	\begin{itemize}
		\item public List<Point> getClusteredCentralPoints(Event event) :
		\begin{description}
			\item This method fetches all locations from the database which belong to the event and calls the cluster algorithm to get it clustered. After that it calls the MidpointAlgo to get the midpoints from the calculated clusters.
			\item @param the event which midpoint should be calculated.
			\item @return central points of the calculated clusters
		\end{description}
	\end{itemize}
	
	

	
	
	\subsubsection{abstract MidPointAlgo}
	
	This is the abstract class with the method to calculate the central of a cluster.

	Methods:
	\begin{itemize}
		\item public Point calculateCentralPoint(CentroidCluster cluster) 
		\begin{description}
			\item This is the abstract method for calculating a clusters central point.
			\item @param the cluster which midpoint should be calculated.
			\item @return the midpoint.
		\end{description}
	\end{itemize}
	%TODO bis hier
	
	
	
	\subsubsection{SimpleCentral extends MidPointAlgo}
	
	This is the class which calculates the arithmetic midpoint. It extends the class MidPointAlgo and overwrites its methods.	
	
	
	
	
	
	
	\subsubsection{ImportantMidPointCentral extends MidPointAlgo}
	
	This class is for calculating the midpoint of a cluster but not every point is weighted the same, especially the midpoint of the cluster is more important than other points. It overwrites the method calculateCentralPoint from its super class MidPointAlgo. 	
	
	

	\subsection{edu.kit.pse.gruppe1.goApp.server.servlet}
	Das Package servlet nimmt auf dem Server alle Anfragen der Clienten entgegen wertet sie aus und leitet diese entsprechend weiter. Eine Beschreibung aller Servlets findet sich \hyperlink{Servlets}{hier}, weil sie thematisch besser in das Kapitel Kommunikation Server \& Client passt.
	\newpage
