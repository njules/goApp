% !TEX root = Entwurf_goApp.tex

\section{JSON-RPC: Kommunikation Server \& Client}
Für die Kommunikation zwischen Server und Client benutzen wir JSON-RPC und das HTTP-Protokoll.

\subsection{JSON}
JSON (JavaScript Object Notation) ist ein Datenformat mit welchem man zwar kompakt, aber dennoch lesbar, Daten codieren und über das HTTP-Protokoll austauschen kann.
Zum codieren und decodieren unserer JSON-Strings haben wir uns dazu entschieden org.JSON zu verwenden.
JSON Beispiel:
{
  "ID": "1"
  "method": "create"	
  "UserID": "1234",
  "GroupName": "KIT Mensa",
}

\subsection{RPC}
RPC (Remote Procedure Call) ermöglicht es Methoden auf einem entfernten Rechner/Server aufzurufen. Wir verwenden es um unser Client-Server-Modell zu implementieren.
Wir rufen vom Client (Android-Smartphone) eine Methode eines Servlets auf dem Server auf. Diese bearbeitet dann die Anfrage und schickt eine Antwort zurück.
Welche Methode aufgerufen werden soll codieren wir in der Anfrage durch Angabe des Methodennamens. Das Servlet kann dadurch den richtigen Methodenaufruf initiieren.
Der Parameter der Methode ist der JSON-String vom Client und der Rückgabewert ist der Antwort-JSON-String, welcher zum Client zurück geschickt wird.


\subsection{HTTP-Protokoll}
Das HTTP-Protokoll beinhaltet verschiedene Anfragemethoden.
Wir verwenden allerdings nur GET und POST, welches die zwei wichtigsten sind.
Wir teilen die aufrufbaren Methoden auf die 2 Anfragemethoden auf:
1. GET-Requests: Anfragen die nur Daten abfragen ohne dabei Daten zu verändern.
2. POST-Requests: Anfragen die Daten bearbeiten/verändern.

Daher gilt, dass man GET-Requests beliebig oft ausführen kann ohne einen Fehler zu erzeugen, da keine Daten verändert werden.

	\subsection{edu.kit.sdqweb.pse.gruppe1.goApp.client.controler.severConnection}
	\subsection{edu.kit.sdqweb.pse.gruppe1.goApp.server.servlet}
	\subsubsection{Status}
	%TODO Jedes Servlet als Subsubsection mit sinnvoller Bennenung.
	\subsubsection{GroupServlet}
	\newpage