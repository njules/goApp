% !TEX root = Entwurf_goApp.tex

\section{JSON-RPC: Kommunikation Server \& Client}
Für die Kommunikation zwischen Server und Client benutzen wir JSON-RPC und das HTTP-Protokoll.

\subsection{JSON}
JSON (JavaScript Object Notation) ist ein Datenformat mit welchem man zwar kompakt, aber dennoch lesbar, Daten codieren und über das HTTP-Protokoll austauschen kann.
Zum codieren und decodieren unserer JSON-Strings haben wir uns dazu entschieden org.JSON zu verwenden.
JSON Beispiel:
{
  "ID": "1"
  "method": "create"	
  "UserID": "1234",
  "GroupName": "KIT Mensa",
}

\subsection{RPC}
RPC (Remote Procedure Call) ermöglicht es Methoden auf einem entfernten Rechner/Server aufzurufen. Wir verwenden es um unser Client-Server-Modell zu implementieren.
Wir rufen vom Client (Android-Smartphone) eine Methode eines Servlets auf dem Server auf. Diese bearbeitet dann die Anfrage und schickt eine Antwort zurück.
Welche Methode aufgerufen werden soll codieren wir in der Anfrage durch Angabe des Methodennamens. Das Servlet kann dadurch den richtigen Methodenaufruf initiieren.
Der Parameter der Methode ist der JSON-String vom Client und der Rückgabewert ist der Antwort-JSON-String, welcher zum Client zurück geschickt wird.


\subsection{HTTP-Protokoll}
Das HTTP-Protokoll beinhaltet verschiedene Anfragemethoden.
Wir verwenden allerdings nur GET und POST, welches die zwei wichtigsten sind.
Wir teilen die aufrufbaren Methoden auf die 2 Anfragemethoden auf:
1. GET-Requests: Anfragen die nur Daten abfragen ohne dabei Daten zu verändern.
2. POST-Requests: Anfragen die Daten bearbeiten/verändern.

Daher gilt, dass man GET-Requests beliebig oft ausführen kann ohne einen Fehler zu erzeugen, da keine Daten verändert werden.


\hypertarget{ServerConnection}{}
	\subsection{edu.kit.sdqweb.pse.gruppe1.goApp.client.controler.ServerConnection}
	 	Nachfolgend werden die Klassen des Paketes ServerConnection näher erläutert.
 	Dieses stellt die Schnittstelle zum Server dar. Das Paket benutzt von org.json die Klassen JSONObject und JSONArray.
 	JSONObject repräsentiert einen JSON-String, hat einen Konstruktor dem man einen JSON-String übergeben kann und besitzt die Methoden get und put um Elemente aus dem JSON-String auszulesen bzw. Elemente einem JSON-String hinzuzufügen. 
 	Ein JSONArray ist eine Menge von Objekten des gleichen Typs welche zu einem JSONObject gehören.
 	
	\subsubsection{public class HTTPConnection}
If a service wants to communicate with the server, the service has to create a HTTPConnection object and must call the sendGet/PostRequest method.
For later requests the service can use the same HTTPConnection object.
\newline Methods:
\begin{itemize}
	\item public HTTPConnection(String nameOfServlet)
		\begin{description}
		\item Constructor which expects the name of the servlet.
	 \item @param nameOfServlet The name of the servlet, the service wants to communicate with.
		\end{description}
		
		\item public void sendGetRequest(String JSON)
		\begin{description}
\item Method that handles a get request.
	 \item @param JSON The String which should be send to the server.
	 \item @return The JSONObject of the JSON-String which was sent back from the Server
		\end{description}
		
		\item public void sendPostRequest(String JSON)
		\begin{description}
\item Method that handles a post request.
	 \item @param JSON The String which should be send to the server.
	 \item @return The JSONObject of the JSON-String which was sent back from the Server
		\end{description}
	\end{itemize}
	
	\subsubsection{public enum JSONParameter}
Enumeration with all possible parameter-types for the JSON-strings.
	\newline Enum literals:
	\begin{itemize}
	
	\item ID
	\begin{description}
	\item ID of the request.
	\end{description}
	
	 \item ErrorCode
	 \begin{description}
	 \item Error code which is 0 if no error occurred.
	 \end{description}
	 
	 \item UserID
	 \begin{description}
	 \item ID of an user.
	 \end{description}
	 
     \item GroupID
     \begin{description}
	 \item ID of a group.
\end{description}

	\item EventID
	\begin{description}
	\item ID of an event.
\end{description}
	 
	\item UserName
	\begin{description}
	\item Name of an user.
\end{description}

	\item GroupName	
	\begin{description}
	\item Name of a group.
\end{description}
	 
	\item EventName
	\begin{description}
	\item Name of an event.
\end{description}
	 
	\item Method 
	\begin{description}
	\item The name of the method which should be executed on the server. For example the create method of the GroupServlet.
\end{description}
	 
	%TODO
	\item and further ones
\end{itemize}

Methods:
\begin{itemize}
	\item public String toString()
		\begin{description}
		\item Gives the corresponding name to an enum literal. Normally the enum literal name.
		\item @return The corresponding name.
		\end{description}
		
	\item public static JSONParameter fromString(String s)
		\begin{description}
		\item Gives the corresponding enum literal to a string.
	 \item @param s The string to the searched enum literal.
     \item @return The corresponding enum literal.
		\end{description}
\end{itemize}
	\subsection{edu.kit.sdqweb.pse.gruppe1.goApp.server.servlet}
	\subsubsection{Status}
	%TODO Jedes Servlet als Subsubsection mit sinnvoller Bennenung.
	\subsubsection{GroupServlet}
	\newpage