\documentclass{scrartcl}
\usepackage[utf8]{inputenc}
\usepackage{verbatim}
\usepackage{enumitem}
\usepackage{graphicx}
\usepackage[ngerman]{babel}
\usepackage{hyperref}
\usepackage{xcolor}
\hypersetup{
    colorlinks,
    linkcolor={blue!65!black},
    citecolor={blue!50!black},
    urlcolor={blue!80!black}
}
\title{goApp Entwurf}
\author{Jörn Kussmaul, Katharina Riesterer, Julian Neubert,\\ Jonas Walter, Tobias Ohlsson, Eva-Maria Neumann}
\begin{document}
	\maketitle
	\newpage
	\tableofcontents
	\newpage

	\section{Entwurfsentscheidungen}
	\subsection{Einleitung}
	Im folgenden Dokument stellen wir ...
	\subsection{Paketstruktur}
	 \includegraphics[width=1\textwidth]{Packages.png}
	\subsection{Server}
	\subsubsection{Model}
	\subsubsection{DatabaseAdapter}
	\subsubsection{Database}
	\subsubsection{Clustering-Algorithm}
	\subsubsection{Servlets}
	\subsection{Client}
	\subsubsection{ServerAdapter}
	\subsubsection{Services}
	\subsubsection{Activities}
	\subsubsection{Model}
	\newpage

	\section{JSON-RPC: Kommunikation Server \& Client}
	\subsection{Motivation}
	\subsection{HTTP-Protokoll}
	\subsection{ServiceAdapter}
	\subsection{Servlets}
	\subsubsection{Status}
	%TODO Jedes Servlet als Subsubsection mit sinnvoller Bennenung.
	\subsubsection{GroupServlet}
	\newpage

	\section{Server}
	%TODO Jedes Package (mit Ausnahme der Servlets) als Subsection mit sinnvoller Bennenung und in diesen alle Klassen beschreiben.
	\subsection{Model}
	\subsection{DatabaseAdapter}
	\subsection{Database}
	\subsection{Clustering-Algorithm}
	\newpage

	\section{Client} 
	%TODO Jedes Package als subsection mit sinnvoller Bennenung und in diesen als subssubsection alle Klassen beschreiben.
	\subsection{Services}
	\subsection{Activities}
	\subsection{Model}
	%TODO bei Datenbank auf Clienten die entsprechenden Packages eintragen.
	\newpage

	\section{Sequenzdiagramme}
	\newpage

	\section{Änderungen zum Pflichtenheft}
	%TODO realisierte Wunschkriterien, gestrichene Kriterien (möglichst klein halten)
	\subsection{Zusätzliche Funktionalität gegenüber dem Pflichtenheft}
	\subsubsection{GO-Button}
		Benutzer müssen wenn Sie an einem Termin teilnehmen den Go-Button drücken sobald Sie sich auf den Weg machen um vom Cluster-Algorithmus erfasst zu werden. 
		\newline
		Vorteile:
		\begin{itemize}
		\item Teilnehmer die sich noch Zuhause (oder andersweitig) befinden verzerren nicht den Mittelpunkt der Gruppe.
		\item Teilnehmer können sehen ob und welche anderen Benutzer bereits unterwegs sind und ihre Entscheidung davon abhängig machen.
		\end{itemize}
	\subsection{gestrichene Funktionalität gegenüber dem Pflichtenheft}
	\newpage
	
	\section{Klassendiagramm}	
\end{document}
