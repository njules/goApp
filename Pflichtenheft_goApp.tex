\documentclass{scrartcl}
\usepackage[utf8]{inputenc}
\title{Android Go App - Pflichtenheft}

\begin{document}
	\maketitle
	\newpage
	
	\tableofcontents
	\newpage
	
	\section{Zielbestimmung}
	\subsection{Musskriterien}
		\begin{itemize}
		\item Anmeldung mit Googleservices
		\item Ein User kann Mitglied mehrerer Gruppen sein und kann über diese Informationen abrufen.
		\item Gruppenverwaltung
			\begin{itemize}
			\item Erstellen neuer Gruppen.
			\item Suche bestehender Gruppen.
			\item Gründer, User Rollenstruktur.
			\item Mitglieder in die Gruppe aufnehmen, aus der Gruppe entfernen.
			\end{itemize}
		\item Terminverwaltung
			\begin{itemize}
			\item Termine können mit Zeit, Ort, Name innerhalb einer Gruppe erstellt werden.
			\item User können bei Terminen zu-/absagen.
			\item User werden an ihren Termin erinnert.
			\end{itemize}
		\item Ein User kann seinen eigenen Standort und den Gruppenmittelpunkt abrufen.	
		\end{itemize}
	\subsection{Wunschkriterien}
		\begin{itemize}
		\item "Bin Los" und "Bin da"
			\begin{itemize}
			\item User können per Button signalisieren ob sie bereits zum Treffpunkt unterwegs sind oder diesen sogar schon erreicht haben.
			\item Nur User die Unterwegs sind werden für den Gruppenmittelpunkt beachtet.
			\item User die den Treffpunkt erreicht haben werden für die Berechnung der Gruppenmitte stärker fokussiert.
			\end{itemize}
		\item Erinnerung an Termin auch bei geschlossener App.
		\item Termine können gelöscht und geändert werden
		\item Statistiken über Teilnahme der User speichern.
		\item QR-Code zum finden von Gruppen
		\item Sprachwahl
		\end{itemize}
	\subsection{Abgrenzungskriterien}
		\begin{itemize}
		\item Kein Chat.
		\item User können nicht die Standorte andere User abfragen.
		\end{itemize}
\end{document}
