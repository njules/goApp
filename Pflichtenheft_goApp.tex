\documentclass{scrartcl}
\usepackage[utf8]{inputenc}
\usepackage[nonumberlist]{glossaries}
\usepackage{verbatim}
\usepackage{enumitem}
\usepackage{graphicx}
\usepackage[ngerman]{babel}
\title{Android Go App - Pflichtenheft}

\begin{document}
	\maketitle
	\newpage
	
	\tableofcontents
	\newpage
	
	\section{Zielbestimmung}
	\subsection{Musskriterien}
		\begin{itemize}
		\item Anmeldung mit Googleservices
		\item Ein User kann Mitglied mehrerer Gruppen sein und kann über diese Informationen abrufen.
		\item Gruppenverwaltung
			\begin{itemize}
			\item Erstellen neuer Gruppen.
			\item Suche bestehender Gruppen.
			\item Gründer, User Rollenstruktur.
			\item Mitglieder in die Gruppe aufnehmen, aus der Gruppe entfernen.
			\end{itemize}
		\item Terminverwaltung
			\begin{itemize}
			\item Termine können mit Zeit, Ort, Name innerhalb einer Gruppe erstellt werden.
			\item User können bei Terminen zu-/absagen.
			\item User werden an ihren Termin erinnert.
			\end{itemize}
		\item Ein User kann seinen eigenen Standort und den Gruppenmittelpunkt abrufen.	
		\end{itemize}
	\subsection{Wunschkriterien}
		\begin{itemize}
		\item \grqq{}Bin Los\grqq{} und \grqq{}Bin da\grqq{}
			\begin{itemize}
			\item User können per Button signalisieren ob sie bereits zum Treffpunkt unterwegs sind oder diesen sogar schon erreicht haben.
			\item Nur User die Unterwegs sind werden für den Gruppenmittelpunkt beachtet.
			\item User die den Treffpunkt erreicht haben werden für die Berechnung der Gruppenmitte stärker fokussiert.
			\end{itemize}
		\item Erinnerung an Termin auch bei geschlossener App.
		\item Termine können gelöscht und geändert werden
		\item Statistiken über Teilnahme der User speichern.
		\item QR-Code zum finden von Gruppen
		\item Sprachwahl
		\end{itemize}
	\subsection{Abgrenzungskriterien}
		\begin{itemize}
		\item Kein Chat.
		\item User können nicht die Standorte andere User abfragen.
		\end{itemize}

	\newpage

	

	\section{Produkteinsatz}
	\begin{itemize}	        
		\item Das Produkt soll das spontane Organisieren von Treffen unterstützen. Dazu soll es dem Nutzer möglich sein in einer Gruppe einen Termin zu erstellen, bzw. bei Terminen zu- und abzusagen.
		\item Kurz vor dem Termin wird dann der mittlere Standort der Gruppe angezeigt, um das Finden der anderen Gruppenmitglieder zu erleichtern.
		\end{itemize}
	\subsection{Andwendungsbereich}
		\begin{itemize}	        
		\item Spontane Terminvereinbarung 
		\end{itemize}
		\subsection{Zielgruppe}
		\begin{itemize}	        
		\item Alle Studierende und Mitarbeiter am KIT
		\end{itemize}
		\subsection{Betriebsbedingung}
		\begin{itemize}	        
		\item App kann überall eingesetzt werden
		\end{itemize}

	\newpage
	

	\section{Produktumgebung}
	\begin{itemize}	        
		\item Eine Client-Server-Architektur
		\item Der Client ist eine Android-App
		\item Die App läuft parallel mit anderen Apps auf einem Android-Smartphone ohne mit diesen zu kommunizieren
		\end{itemize}
\subsection{Software}
		\begin{itemize}	        
		\item Client: Android ab 4.0
		\item Server: Tomcat 8
		\end{itemize}	
\subsection{Orgware}
		\begin{itemize}	        
		\item Internetverbindung und GPS
		\end{itemize}
		\subsection{Produktschnittstelle}
		\begin{itemize}	        
		\item Google Maps
		\end{itemize}
	\newpage


	\section{Funktionale Anforderungen}
		

		\subsection{Accountverwaltung}
		\begin{itemize}
		\item[FA10] Anmeldung eines Benutzers in der App über Googleservices.
		\item[FA20] Ersterfassung und Änderung des Benutzernamens.
		\end{itemize}
		
		\subsection{Gruppenverwaltung}
		\begin{itemize}
		\item[FA30] Jeder Benutzer kann eine Gruppe erstellen.
		\item[FA35] Ersterfassung und spätere Änderung des Gruppennamens durch den Gruppengründer.
		\item[FA40] Gruppen können über den Gruppennamen gesucht werden.
		\item[FA45] Benutzer können Beitrittsanfragen an eine Gruppe senden.
		\item[FA50]Der Gruppengründer kann Betrittsanfragen zu der Gruppe verwalten:
			\begin{itemize}
			\item Anfragen bestätigen, wodurch der Anfragende Mitglied der Gruppe wird und die Anfrage gelöscht wird.
			\end{itemize}
			\begin{itemize}
			\item Anfragen ablehnen, wodurch die Anfrage gelöscht wird.
			\end{itemize}
			\begin{itemize}
			\item Anfragen ignorieren, wodurch die Anfrage bestehen und sichtbar bleibt.
			\end{itemize}
		 \item[FA60] Der Gruppengründer kann bestehende Mitglieder aus der Gruppe entfernen.
		 \item[FA70] Die Gruppe kann durch den Gruppengründer gelöscht werden.
		 \item[FA80] Mitglieder der Gruppen, ausgenommen des Gruppengründers, können die Gruppe verlassen.
		 \item[FA90W] Der Gruppengründer kann seinen Status als Gruppengründer an ein Mitglied übergeben.
		\end{itemize}
		
		\subsection{Terminverwaltung}
		\begin{itemize}
		\item[FA100] Jedes Mitglied einer Gruppe kann einen Termin erstellen.
		\item[FA110] Der Terminersteller kann den von ihm erstellten Termin verwalten:
			\begin{itemize}
			\item Ersterfassung und Änderung des Terminzeitpunktes.
			\end{itemize}
			\begin{itemize}
			\item Ersterfassung und Änderung des Terminortes.
			\end{itemize}
			\begin{itemize}
			\item Ersterfassung und Änderung des Terminnamens.
			\end{itemize}
			
		\item[FA120] Der Termin wird jedem Gruppenmitglied angezeigt.
		\item[FA130] Jedes Gruppenmitglied kann einem Termin zu- oder absagen.
		\item[FA140] Jeder Teilnehmer wird vor Beginn des Termins benachrichtigt werden.
		
		\end{itemize}
		
		
		
		
	\newpage


	\section{Produktdaten}
	%TODO: XXX jeweils durch Zahlen ersetzen
	\begin{itemize}
		\item [D10] \textit{Nutzerdaten:}
		Es gibt maximal XXX Nutzer. Für jeden Nutzer werden die folgenden Daten gespeichert bzw. erfasst:
		\begin{itemize}
			\item Benutzername
			\newline Ein eindeutiger Name für den Benutzer, der aus maximal XXX Zeichen besteht.
			\item Anzeigename
			\newline Ein frei gewählter Name, der den anderen Nutzern der App angezeigt wird. Er besteht aus maximal XXX Buchstaben. %auch Zahlen?
			\item Standort
			\newline Der aktuelle Standort der Person, damit der gemittelte Standort berechnet werden kann.
			\item Status
			\newline 
			\item GoogleServices Informationen
			\newline 
		\end{itemize}
		
		\item [D20] \textit{Gruppendaten:}
		Es kann maximal XXX Gruppen geben. Eine Gruppe hat die folgenden Eigenschaften:
		\begin{itemize}
			\item Name
			\newline Ein eindeutiger Name der Gruppe, der maximal XXX Buchstaben/Zeichen lang sein kann.
			\item Teilnehmer
			\newline In einer Gruppe können maximal XXX Teilnehmer sein, als diese werden Nutzer mit ihrem Nutzernamen identifiziert.
			\item Rechte
			\newline
			\item Termine
			\newline Eine Gruppe kann maximal XXX Termine haben. Diese werden noch näher erläutert.
		\end{itemize}
		
		\item [D30] \textit{Termindaten:}
		In der App gibt es maximal XXX aktive Termine gleichzeitig. Jeder Termin beinhaltet die folgenden Daten
		\begin{itemize}
			\item Name
			\newline Der Name des Termins.
			\item Ziel
			\newline Das Ziel ist der Standort, wo das Treffen stattfinden soll.
			\item Teilnehmer
			\newline An einem Termin können maximal XXX Teilnehmer beteiligt sein. 
			\item Gruppe
			\newline Der Termin gehört zu einer bestimmten Gruppe
			\item Serverzeit
			\newline
		\end{itemize}
	\end{itemize}

	\newpage


	\section{Qualitäts-Anforderungen}
	\begin{itemize}[nosep] 
	\item[QA10] Die App soll auf jede Anfrage in durchschnittlich 5 Sekunden reagieren.
	\item[QA20] Die App fährt in durchschnittlich 10 Sekunden hoch.
	\item[QA30] Jede Funktion der App ist mit höchstens 5 Eingaben vom Startbildschirm zu erreichen.
	\item[QA40] Die App fällt nicht aus, auch bei widrigen Bedingungen (kein Netzwerkverbindung, Standortungenauigkeit).
	\item[QA50] Unterstützt bis zu 20 User pro Gruppe.
	\item[QA60] Jeder User kann in bis zu 10 Gruppen Mitglied sein.
	\end{itemize}

	\newpage


	\section{Globale Testfälle}

	\newpage


	\section{Systemmodelle}
	
	\subsection{Anwendungsfälle}
	\subsubsection{Nutzer Registrieren}
	\begin{description}
		\item \textbf{Titel:}
		\begin{description}
			\item Nutzer Registrieren
		\end{description}
		\item \textbf{Kurzbeschreibung:}
		\begin{description}
			\item Der Nutzer registriert sich mit seinem Google Play Account
		\end{description}
		\item \textbf{Aktoren:}
		\begin{description}
			\item Nutzer 
		\end{description}
		\item \textbf{Vorbedingungen:}
		\begin{description}
			\item Der Nutzer besitzt einen Google Play Account
			\item Mit diesem Google Play Account ist noch kein Konto verknüpft
		\end{description}
		\item \textbf{Ablauf:} \newline Für den Nutzer wird ein Konto auf der Datenbank erstellt und mit seinem Google Play Account verknüpft.
		\item \textbf{Ergebnis:} \newline Der Nutzer besitzt ein Konto und kann somit zu Gruppen hinzugefügt werden oder selber Gruppen erstellen.
	\end{description}

	\newpage
	
	\subsubsection{Gruppe erstellen}
	\begin{description}
		\item \textbf{Titel:}
		\begin{description}
			\item Gruppe erstellen
		\end{description}
		\item \textbf{Kurzbeschreibung:}
		\begin{description}
			\item Der Nutzer Gründet eine Gruppe
		\end{description}
		\item \textbf{Aktoren:}
		\begin{description}
			\item Nutzer 
		\end{description}
		\item \textbf{Vorbedingungen:}
		\begin{description}
			\item Der Nutzer hat sich zuvor erfolgreich registriert.
		\end{description}
		\item \textbf{Ablauf:} \newline Es wird eine Gruppe erstellt, für die der erstellende Nutzer als Gründer eingetragen wird. Der Nutzer muss einen Namen für seine Gruppe wählen. Damit erhält der Nutzer auch alle Rechte, die dem Gruppengründer zugestanden werden.
		\item \textbf{Ergebnis:} \newline Der Nutzer ist Gründer einer Gruppe und kann diese im Rahmen seiner Rechte verwalten.
	\end{description}
	
	\newpage
	
	\subsection{Anwendungsfalldiagramm}
	\begin{figure}[h]
	\includegraphics[width=\textwidth]{goApp_useCase}
	\end{figure}

	\newpage
	
	\subsection{Szenarien}
	\subsubsection{Szenario 1}
	Alice, Bob und Carol gehen oft gemeinsam in der Mensa essen. Zur besseren Koordination haben sie die App auf ihren Android-Smartphones installiert.
	Nun startet jeder bei sich die App und gibt, nachdem er sich über sein Google Konto angemeldet hat, seinen Namen ein.
	Alice gründet nun eine neue Gruppe mit dem Namen ABC-Mensa. Danach geben Bob und Carol in das Suchfeld ABC-Mensa ein und klicken auf die entsprechende Gruppe.
	Es erscheint nun auf dem Display die Nachfrage ob sie der Gruppe beitreten wollen. Sobald sie dies bestätigt haben klickt Alice auf die Gruppe,
	sieht die Namen Bob und Carol als Bewerber, und fügt beide hinzu.
	Bob bekommt nun Hunger, klickt auf die Gruppe und erstellt einen neuen Termin mit dem Namen Mensa, dem entsprechenden Datum, der Uhrzeit 12:30 und dem Ort der Mensa.
	Kurze Zeit darauf schauen Alice und Carol auf ihr Smartphone, sehen den Termin und klicken beide auf \glqq{}Teilnehmen\grqq{}.
	Um 12:15 erscheint auf allen Smartphones die Benachrichtigung, dass in 15 Minuten der Termin ansteht.
	Alle gehen kurz daraufhin los und können in der App sowohl ihren als auch den mittleren Gruppenstandort sehen.
	Nachdem der Termin vorbei ist, wird er automatisch gelöscht.
	\newline
	\newline
	\begin{figure}[h]
	\includegraphics[width=\textwidth]{Szenario1}
	\end{figure}
	
	\newpage
	
	\subsubsection{Szenario 2}
	Die Personen Alice, Bob, Carol, Dave und Eve sind in einer gemeinsamen Gruppe.
	Alice will um 12:00 Uhr in die Mensa gehen. Daher erstellt Alice in der Gruppe einen neuen Termin \glqq{}Mensa 12:00\grqq{}.
	Alle anderen Gruppenmitglieder sehen den neuen Termin. Bob und Carol wollen auch um 12:00 in die Mensa gehen und klicken auf \glqq{}Teilnehmen\grqq{}.
	Dave und Eve müssen um 12:00 Uhr noch etwas erledigen und können deshalb erst später in die Mensa, sie klicken deshalb auf \glqq{}Nicht Teilnehmen\grqq{}.
	Da sie auf \glqq{}Nicht Teilnehmen\grqq{} geklickt haben, ist der Termin für sie nicht mehr sichtbar.
	Dave erstellt später einen neuen Termin \glqq{}Mensa 12:30\grqq{} in der Gruppe.
	Um 12:30 hat Eve auch Zeit und klickt auf \glqq{}Teilnehmen\grqq{}.
	Da Alice, Bob und Carol schon früher in die Mensa gehen klicken sie auf \glqq{}Nicht Teilnehmen\grqq{}.
	Um 11:45 werden Alice, Bob und Carol an ihren Termin erinnert und um 12:15 Dave und Eve.
	Der mittlere Gruppenstandort bezieht sich immer ausschließlich auf die Personen, die an dem Termin teilnehmen.
	So sieht Alice den mittleren Standort von sich, Bob und Carol.	
	\newline
	\newline
	\newline
	\begin{figure}[h]
	\includegraphics[width=\textwidth]{Szenario2}
	\end{figure}
	\newpage

	\subsection{GUI}
	\begin{figure}[h]
	\includegraphics[width=.5\textwidth]{GUI_Start.jpg}
	\includegraphics[width=.5\textwidth]{GUI_changeName.jpg}
	\caption{Die Startseite zeigt den Namen des Benutzers und bietet eine übersicht über alle Gruppen in denen der Benutzer Mitglied ist. Außerdem lässt sich hier der eigene Anzeigename ändern.}
 	\end{figure}

	\newpage
	\begin{figure}[h]
	\includegraphics[width=.5\textwidth]{GUI_NeueGruppe.jpg}
	\includegraphics[width=.5\textwidth]{GUI_GruppeNeuBest.jpg}
	\caption{Die Gruppensuche zeigt dem Benutzer alle Gruppen an deren Name mit seiner Eingabe startet, zusätzlich kann der Benutzer hier eine neue Gruppe erstellen.}
 	\end{figure}

	\newpage
	\begin{figure}[h]
	\centering
	\includegraphics[width=.5\textwidth]{GUI_Gruppe.jpg}
	\caption{In einer Gruppe bekommt ein  Mitglied die Termine der Gruppe angezeigt und kann bei diesen zu- oder absagen, mit einem wischen kann zwischen allen aktuellen Terminen gewechselt werden. Es kann ebenfalls ein eigener Termin erstellt und weitere Details sowohl über die Gruppe als auch die Termine abgerufen werden.}
 	\end{figure}

	\newpage
	\begin{figure}[h]
	\includegraphics[width=.5\textwidth]{GUI_GruppeInfoNormal.jpg}
	\includegraphics[width=.5\textwidth]{GUI_GruppeInfoGruender.jpg}
	\caption{In der Gruppeninformationsansicht sehen Mitglieder alle anderen Mitglieder der Gruppe und sie können die Gruppe verlassen. Der Gründer der Gruppe kann zusätzlich einzelne Mitglieder aus der Gruppe entfernen und Beitrittsanfragen bearbeiten, anstatt die Gruppe zu verlassen kann er die Gruppe löschen.}
 	\end{figure}

	\newpage
	\begin{figure}[h]
	\centering
	\includegraphics[width=.5\textwidth]{GUI_NeuerTermin.jpg}
	\caption{Damit ein Mitglied einen neuen Termin erstellen kann muss es dem Termin einen Namen geben und ihn mit Datum, Uhrzeit und einem Ort versehen.}
 	\end{figure}

	\newpage
	\begin{figure}[h]
	\centering
	\includegraphics[width=.5\textwidth]{GUI_Termin.jpg}
	\caption{In der Terminansicht erhält ein Mitglied alle Informationen über einen Termin, darunter sind eine Teilnehmerliste und eine Karte die den Treffpunkt und den eigenen Standort anzeigt. Sollte der Termin kurz bevorstehen wird zusätzlich der gemittelte Gruppenstandort angezeigt.}
 	\end{figure}
	

	\begin{comment}
	\makeglossary
	
\newglossaryentry{Client-Server-Architektur} 
	{
	name=Client-Server-Architektur,
	description={Konzept zur Verteilung von Aufgaben in Netzwerken. }
	}
	
	\newglossaryentry{Server} 
	{
	name=Server
	description={Ein Server kommuniziert mit einem Client, um Aufgaben/Anfragen vom Client zu bearbeiten. }
	}
	
	\newglossaryentry{Client} 
	{
	name=Client
	description={Ein Client kann Aufgaben/Anfragen an einen Server schicken. }
	}	
	
	\newglossaryentry{Tomcat} 
	{
	name=Tomcat
	description={Tomcat führt in Java geschriebene Web-Anwendungen auf Servern aus. }
	}	
	
	\newglossaryentry{Android} 
	{
	name=Android
	description={Android ist ein weitverbreitetes Betriebssystem für Smartphones, welches von Google entwickelt wurde.}
	}	

	\newglossaryentry{Gruppengründer} 
	{
	name=Gruppengründer,
	description={Bezeichnet den Status des Mitglieds, das die Gruppe gegründet hat. }
	}
	
	\newglossaryentry{Mitglied}
	{
	name=Miglied,
	description={Benutzer der einer Gruppe angehört.}
	}
	
	\newglossaryentry{Teilnehmer}
	{
	name=Teilnehmer,
	description={Ein Mitglied, das an einem Termin teilnimmt}
	}
	\end{comment}
	
	
\end{document}
