\documentclass{scrartcl}
\usepackage[utf8]{inputenc}
\usepackage[nonumberlist]{glossaries}
\usepackage{verbatim}
\usepackage{enumitem}
\usepackage{graphicx}
\title{Android Go App - Pflichtenheft}

\begin{document}
	\maketitle
	\newpage
	
	\tableofcontents
	\newpage
	
	\section{Zielbestimmung}
	\subsection{Musskriterien}
		\begin{itemize}
		\item Anmeldung mit Googleservices
		\item Ein User kann Mitglied mehrerer Gruppen sein und kann über diese Informationen abrufen.
		\item Gruppenverwaltung
			\begin{itemize}
			\item Erstellen neuer Gruppen.
			\item Suche bestehender Gruppen.
			\item Gründer, User Rollenstruktur.
			\item Mitglieder in die Gruppe aufnehmen, aus der Gruppe entfernen.
			\end{itemize}
		\item Terminverwaltung
			\begin{itemize}
			\item Termine können mit Zeit, Ort, Name innerhalb einer Gruppe erstellt werden.
			\item User können bei Terminen zu-/absagen.
			\item User werden an ihren Termin erinnert.
			\end{itemize}
		\item Ein User kann seinen eigenen Standort und den Gruppenmittelpunkt abrufen.	
		\end{itemize}
	\subsection{Wunschkriterien}
		\begin{itemize}
		\item "Bin Los" und "Bin da"
			\begin{itemize}
			\item User können per Button signalisieren ob sie bereits zum Treffpunkt unterwegs sind oder diesen sogar schon erreicht haben.
			\item Nur User die Unterwegs sind werden für den Gruppenmittelpunkt beachtet.
			\item User die den Treffpunkt erreicht haben werden für die Berechnung der Gruppenmitte stärker fokussiert.
			\end{itemize}
		\item Erinnerung an Termin auch bei geschlossener App.
		\item Termine können gelöscht und geändert werden
		\item Statistiken über Teilnahme der User speichern.
		\item QR-Code zum finden von Gruppen
		\item Sprachwahl
		\end{itemize}
	\subsection{Abgrenzungskriterien}
		\begin{itemize}
		\item Kein Chat.
		\item User können nicht die Standorte andere User abfragen.
		\end{itemize}

	\newpage

	

	\section{Produkteinsatz}
	\begin{itemize}	        
		\item Das Produkt soll das spontane Organisieren von Treffen unterstützen. Dazu soll es dem Nutzer möglich sein in einer Gruppe einen Termin zu erstellen, bzw. bei Terminen zu- und abzusagen.
		\item Kurz vor dem Termin wird dann der mittlere Standort der Gruppe angezeigt, um das Finden der anderen Gruppenmitglieder zu erleichtern.
		\end{itemize}
	\subsection{Andwendungsbereich}
		\begin{itemize}	        
		\item Spontane Terminvereinbarung 
		\end{itemize}
		\subsection{Zielgruppe}
		\begin{itemize}	        
		\item Alle Studierende und Mitarbeiter am KIT
		\end{itemize}
		\subsection{Betriebsbedingung}
		\begin{itemize}	        
		\item App kann überall eingesetzt werden
		\end{itemize}

	\newpage
	

	\section{Produktumgebung}
	\begin{itemize}	        
		\item Eine Client-Server-Architektur
		\item Der Client ist eine Android-App
		\item Die App läuft parallel mit anderen Apps auf einem Android-Smartphone ohne mit diesen zu kommunizieren
		\end{itemize}
\subsection{Software}
		\begin{itemize}	        
		\item Client: Android ab 4.0
		\item Server: Tomcat 8
		\end{itemize}	
\subsection{Orgware}
		\begin{itemize}	        
		\item Internetverbindung und GPS
		\end{itemize}
		\subsection{Produktschnittstelle}
		\begin{itemize}	        
		\item Google Maps
		\end{itemize}
	\newpage


	\section{Funktionale Anforderungen}
		

		\subsection{Accountverwaltung}
		\begin{itemize}
		\item[FA10] Anmeldung eines Benutzers in der App über Googleservices.
		\item[FA20] Ersterfassung und Änderung des Benutzernamens.
		\end{itemize}
		
		\subsection{Gruppenverwaltung}
		\begin{itemize}
		\item[FA30] Jeder Benutzer kann eine Gruppe erstellen.
		\item[FA35] Ersterfassung und spätere Änderung des Gruppennamens durch den Gruppengründer.
		\item[FA40] Gruppen können über den Gruppennamen gesucht werden.
		\item[FA45] Benutzer können Beitrittsanfragen an eine Gruppen senden.
		\item[FA50]Der Gruppengründer kann Betrittsanfragen zu der Gruppe verwalten:
			\begin{itemize}
			\item Anfragen bestätigen, wodurch der Anfragende Mitglied der Gruppe wird und die Anfrage gelöscht wird.
			\end{itemize}
			\begin{itemize}
			\item Anfragen ablehnen, wodurch die Anfrage gelöscht wird.
			\end{itemize}
			\begin{itemize}
			\item Anfragen ignorieren, wodurch die Anfrage bestehen und sichtbar bleibt.
			\end{itemize}
		 \item[FA60] Der Gruppengründer kann bestehende Mitglieger aus der Gruppe entfernen.
		 \item[FA70] Die Gruppe kann durch den Gruppengründer gelöscht werden.
		 \item[FA80] Mitglieder der Gruppen, ausgenommen des Gruppengründers, können die Gruppe verlassen.
		 \item[FA90W] Der Gruppengründer kann seinen Status als Gruppengründer an ein Mitglied übergeben.
		\end{itemize}
		
		\subsection{Terminverwaltung}
		\begin{itemize}
		\item[FA100] Jedes Mitglied einer Gruppe kann einen Termin erstellen.
		\item[FA110] Der Terminersteller kann den von ihm erstellten Termin verwalten:
			\begin{itemize}
			\item Ersterfassung und Änderung des Terminzeitpunktes.
			\end{itemize}
			\begin{itemize}
			\item Ersterfassung und Änderung des Terminortes.
			\end{itemize}
			\begin{itemize}
			\item Ersterfassung und Änderung des Terminnamens.
			\end{itemize}
			
		\item[FA120] Der Termin wird jedem Gruppenmitglied angezeigt.
		\item[FA130] Jedes Gruppenmitglied kann einem Termin zu- oder absagen.
		\item[FA140] Jeder Teilnehmer wird vor Beginn des Termins benachrichtigt werden.
		
		\end{itemize}
		
		
		
		
	\newpage


	\section{Produktdaten}
	%TODO: XXX jeweils durch Zahlen ersetzen
	\begin{itemize}
		\item [D10] \textit{Nutzerdaten:}
		Es gibt maximal XXX Nutzer. Für jeden Nutzer werden die folgenden Daten gespeichert bzw. erfasst:
		\begin{itemize}
			\item Benutzername
			\newline Ein eindeutiger Name für den Benutzer, der aus maximal XXX Zeichen besteht.
			\item Anzeigename
			\newline Ein frei gewählter Name, der den anderen Nutzern der App angezeigt wird. Er besteht aus maximal XXX Buchstaben. %auch Zahlen?
			\item Standort
			\newline Der aktuelle Standort der Person, damit der gemittelte Standort berechnet werden kann.
			\item Status
			\newline 
			\item GoogleServices Informationen
			\newline 
		\end{itemize}
		
		\item [D20] \textit{Gruppendaten:}
		Es kann maximal XXX Gruppen geben. Eine Gruppe hat die folgenden Eigenschaften:
		\begin{itemize}
			\item Name
			\newline Ein eindeutiger Name der Gruppe, der maximal XXX Buchstaben/Zeichen lang sein kann.
			\item Teilnehmer
			\newline In einer Gruppe können maximal XXX Teilnehmer sein, als diese werden Nutzer mit ihrem Nutzernamen identifiziert.
			\item Rechte
			\newline
			\item Termine
			\newline Eine Gruppe kann maximal XXX Termine haben. Diese werden noch näher erläutert.
		\end{itemize}
		
		\item [D30] \textit{Termindaten:}
		In der App gibt es maximal XXX aktive Termine gleichzeitig. Jeder Termin beinhaltet die folgenden Daten
		\begin{itemize}
			\item Name
			\newline Der Name des Termins.
			\item Ziel
			\newline Das Ziel ist der Standort, wo das Treffen stattfinden soll.
			\item Teilnehmer
			\newline An einem Termin können maximal XXX Teilnehmer beteiligt sein. 
			\item Gruppe
			\newline Der Termin gehört zu einer bestimmten Gruppe
			\item Serverzeit
			\newline
		\end{itemize}
	\end{itemize}

	\newpage


	\section{Qualitäts-Anforderungen}
	\begin{itemize}[nosep] 
	\item[QA10] Die App soll auf jede Anfrage in durchschnittlich 5 Sekunden reagieren.
	\item[QA20] Die App fährt in durchschnittlich 10 Sekunden hoch.
	\item[QA30] Jede Funktion der App ist mit höchstens 5 Eingaben vom Startbildschirm zu erreichen.
	\item[QA40] Die App fällt nicht aus, auch bei widrigen Bedingungen (kein Netzwerkverbinndung, Standortungenauigkeit).
	\item[QA50] Unterstützt bis zu 20 User pro Gruppe.
	\item[QA60] Jeder User kann in bis zu 10 Gruppen Mitglied sein.
	\end{itemize}

	\newpage


	\section{Globale Testfälle}

	\newpage


	\section{Systemmodelle}
	
	\subsection{Anwendungsfälle}
	\subsubsection{Nutzer Registrieren}
	\begin{description}
		\item \textbf{Titel:}
		\begin{description}
			\item Nutzer Registrieren
		\end{description}
		\item \textbf{Kurzbeschreibung:}
		\begin{description}
			\item Der Nutzer registriert sich mit seinem Google Play Account
		\end{description}
		\item \textbf{Aktoren:}
		\begin{description}
			\item Nutzer 
		\end{description}
		\item \textbf{Vorbedingungen:}
		\begin{description}
			\item Der Nutzer besitzt einen Google Play Account
			\item Mit diesem Google Play Account ist noch kein Konto verknüpft
		\end{description}
		\item \textbf{Ablauf:} \newline Für den Nutzer wird ein Konto auf der Datenbank erstellt und mit seinem Google Play Account verknüpft.
		\item \textbf{Ergebnis:} \newline Der Nutzer besitzt ein Konto und kann somit zu Gruppen hinzugefügt werden oder selber Gruppen erstellen.
	\end{description}

	\newpage
	
	\subsubsection{Gruppe erstellen}
	\begin{description}
		\item \textbf{Titel:}
		\begin{description}
			\item Gruppe erstellen
		\end{description}
		\item \textbf{Kurzbeschreibung:}
		\begin{description}
			\item Der Nutzer Gründet eine Gruppe
		\end{description}
		\item \textbf{Aktoren:}
		\begin{description}
			\item Nutzer 
		\end{description}
		\item \textbf{Vorbedingungen:}
		\begin{description}
			\item Der Nutzer hat sich zuvor erfolgreich registriert.
		\end{description}
		\item \textbf{Ablauf:} \newline Es wird eine Gruppe erstellt, für die der erstellende Nutzer als Gründer eingetragen wird. Der Nutzer muss einen Namen für seine Gruppe wählen. Damit erhält der Nutzer auch alle Rechte, die dem Gruppengründer zugestanden werden.
		\item \textbf{Ergebnis:} \newline Der Nutzer ist Gründer einer Gruppe und kann diese im Rahmen seiner Rechte verwalten.
	\end{description}
	
	\newpage
	
	\subsection{Anwendungsfalldiagramm}
	\begin{figure}[h]
	\includegraphics[width=\textwidth]{goApp_useCase}
	\end{figure}

	\newpage
	
	\begin{comment}
	\makeglossary
	
\newglossaryentry{Client-Server-Architektur} 
	{
	name=Client-Server-Architektur,
	description={Konzept zur Verteilung von Aufgaben in Netzwerken. }
	}
	
	\newglossaryentry{Server} 
	{
	name=Server
	description={Ein Server kommuniziert mit einem Client, um Aufgaben/Anfragen vom Client zu bearbeiten. }
	}
	
	\newglossaryentry{Client} 
	{
	name=Client
	description={Ein Client kann Aufgaben/Anfragen an einen Server schicken. }
	}	
	
	\newglossaryentry{Tomcat} 
	{
	name=Tomcat
	description={Tomcat führt in Java geschriebene Web-Anwendungen auf Servern aus. }
	}	
	
	\newglossaryentry{Android} 
	{
	name=Android
	description={Android ist ein weitverbreitetes Betriebssystem für Smartphones, welches von Google entwickelt wurde.}
	}	

	\newglossaryentry{Gruppengründer} 
	{
	name=Gruppengründer,
	description={Bezeichnet den Status des Mitglieds, das die Gruppe gegründet hat. }
	}
	
	\newglossaryentry{Mitglied}
	{
	name=Miglied,
	description={Benutzer der einer Gruppe angehört.}
	}
	
	\newglossaryentry{Teilnehmer}
	{
	name=Teilnehmer,
	description={Ein Mitglied, das an einem Termin teilnimmt}
	}
	\end{comment}
	
	
\end{document}
