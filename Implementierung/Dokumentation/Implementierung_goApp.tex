\documentclass{scrartcl}
\usepackage[utf8]{inputenc}
\usepackage[T1]{fontenc}
\usepackage{verbatim}
\usepackage{enumitem}
\usepackage{graphicx}
\usepackage[ngerman]{babel}
\usepackage{hyperref}
\usepackage{xcolor}
\usepackage{pdfpages}
\setlength{\parindent}{0em} 
\hypersetup{
    colorlinks,
    linkcolor={blue!65!black},
    citecolor={blue!50!black},
    urlcolor={blue!80!black}
}
\title{goApp Implementierung}
\author{Jörn Kussmaul, Katharina Riesterer, Julian Neubert,\\ Jonas Walter, Tobias Ohlsson, Eva-Maria Neumann}
\begin{document}
	\maketitle
	\newpage
	\tableofcontents
	\newpage

	\section{Einleitung}
	Anschluss auf Plichtenheft \& Entwurf
	
	\newpage
	\section{Änderungen am Entwurf}
	Hier jeder (sofort) Eintragen, wenn er was anders macht, wie eigentlich geplant.
	
	
	\subsection{GoServlet und LocationServlet}
	%TODO GoServlet wird gelöscht
	%Paticipate Servlet wird neu geschrieben
	%Paticipate: aus accept und reject wird setStatus
	%Go: setStarted wird in Participate.setStatus mit reingenommen
	%Go: getStarted wird in Event.get integriert
	%TODO LocationServlet ruft immer beide Methoden auf (setGPS und getCluster)
	
	\subsection{ServletUtils}
	-Methoden werden öfters in verschiedenen Servlet genutzt
	-verringern von redundantem Code
	
	\subsection{Clusteralgorithmus}
	Die Clusterfassade bietet nun neben der Methode getClusteredPoints auch die Methode getClusters und getCenter um direkt auf den Clusterer bzw. den Mittelpunktalgorithmus zugreifen zu können. 
	
	\subsection{JSONParameter}
	-untergeordnete Enums hinzugefügt für Fehler und Methoden
	\newpage
	\section{Kriterien}
	
	\subsection{Erfüllte Kriterien}
	\subsubsection{Kontoverwaltung}
	\begin{itemize}
		\item[FA10] Anmeldung eines Benutzers in der App über Google Services
		\item[FA20] Ersterfassung und Änderung des Benutzernamens
	\end{itemize}
	\subsubsection{Gruppenverwaltung}
		\begin{itemize}
			
			\item[FA30] Jeder Benutzer kann eine Gruppe erstellen
			
			\item[FA35] Ersterfassung und spätere Änderung des Gruppennamens durch den Gruppengründer
			
			\item[FA40] Gruppen können über den Gruppennamen gesucht werden
			
			\item[FA45] Benutzer können Beitrittsanfragen an eine Gruppe senden
			
			\item[FA50] Der Gruppengründer kann Beitrittsanfragen der Gruppe verwalten:
			\begin{itemize}
				\item Anfragen bestätigen, wodurch der Anfragende ein Mitglied der Gruppe wird und die Anfrage gelöscht wird
			\end{itemize}
			\begin{itemize}
				\item Anfragen ablehnen, wodurch die Anfrage gelöscht wird
			\end{itemize}
			\begin{itemize}
				\item Anfragen ignorieren, wodurch die Anfrage bestehen und sichtbar bleibt
			\end{itemize}
			
			\item[FA60] Der Gruppengründer kann Mitglieder aus der Gruppe entfernen
			
			\item[FA70] Die Gruppe kann durch den Gruppengründer gelöscht werden
			
			\item[FA80] Mitglieder der Gruppen, ausgenommen des Gruppengründers, können die Gruppe verlassen
			
			\item[FA90] Gruppenmitglieder können sich andere Gruppenmitglieder anzeigen lassen
	
			\end{itemize}
	
	\subsubsection{Qualitative Anforderungen}
	
	\newpage
	\subsection{Nicht erfüllte Kriterien}
		\subsubsection{Kontoverwaltung}
		\begin{itemize}
			\item[WFA15] Der Benutzer kann zwischen Sprachen wählen
		\end{itemize}
		
		\subsubsection{Gruppenverwaltung}
			\begin{itemize}
			
			\item[WFA85] Der Gruppengründer kann seinen Status als Gruppengründer an ein Mitglied übergeben

			
			\item[WFA95] Gruppenmitglieder können die Teilnahmequoten anderer Gruppenmitglieder einsehen
			\end{itemize}

	\subsubsection{Qualitative Anforderungen}
	
	\newpage
		
	\section{ungeordnete Kriterien aus Pflichtenheft kopiert}
	%TODO: Einsortieren, dann Punkt hier löschen
		
			
			
		
		
		\subsection{Terminverwaltung}
		\begin{itemize}
			\hypertarget{FA100}{}
			\item[FA100] Jedes Mitglied einer Gruppe kann einen Termin erstellen
			
			\begin{itemize}
				\item Ersterfassung von Terminnamen, Terminzeit (in der Zukunft) und Terminort
			\end{itemize}
			
			\begin{itemize}
				\item Visuelle Darstellung des gewählten Terminortes
			\end{itemize}
			\hypertarget{WFA105}{}
			\item[WFA105] Der Terminersteller kann den Terminnamen, Terminzeit und Terminort nachträglich ändern oder gegebenenfalls löschen
			\hypertarget{FA110}{}
			\item[FA110] Der Gruppenmittelpunkt wird anhand der Standorte aller Teilnehmer berechnet		
			\hypertarget{FA120}{}
			\item[FA120] Jeder Termin wird jedem Gruppenmitglied angezeigt
			\hypertarget{FA130}{}
			\item[FA130] Jedes Gruppenmitglied kann bei einem Termin zu- oder absagen
			
		\end{itemize}
		
		\subsection{Terminablauf}
		\begin{itemize}
			\hypertarget{FA140}{}
			\item[FA140] Jeder Teilnehmer wird 30 Minuten vor Beginn des Termins benachrichtigt
			\hypertarget{WFA145}{}
			\item[WFA145] Teilnehmer werden auch bei geschlossener App benachrichtigt
			\hypertarget{FA150}{}
			\item[FA150] Jeder Teilnehmer kann den Gruppenmittelpunkt ab 15 Minuten vor Beginn des Termins in einer Karte darstellen lassen
			\hypertarget{WFA155}{}
			\item[WFA155] Vom Gruppenmittelpunkt entfernte Teilgruppen  erhalten einen eigenen Standort 
			\hypertarget{WFA160}{}
			\item[WFA160] Teilnehmer können in der App anderen Teilnehmern mitteilen,
			\begin{itemize}
				\item dass sie bereits unterwegs sind
				\item dass sie zu spät kommen
				\item dass sie schon am Terminort angekommen sind
			\end{itemize}
			\hypertarget{FA170}{}
			\item[FA170] Der Termin löscht sich nach der Dauer des Termins (standardmäßig 60 Minuten)
			\hypertarget{WFA175}{}
			\item[WFA175] Die Dauer des Termins kann individuell festgelegt werden
		\end{itemize}
		
			\subsection{Qualitative Anforderungen}
			Die Qualitativen Anforderungen beziehen sich auf unser Referenzgerät.
			\begin{itemize}
				\item[QA10] Die App soll auf jede Anfrage in durchschnittlich unter 5 Sekunden reagieren
				\item[QA20] Die App fährt in durchschnittlich unter 10 Sekunden hoch
				\item[QA30] Jede Funktion der App ist mit höchstens 5 Eingaben vom Startbildschirm der App zu erreichen
				\item[QA40] Bei 99\% aller Anwendungen werden Fehler abgefangen und führen nicht zum Absturz der App
				\item[QA50] Unterstützt bis zu 50 Benutzer pro Gruppe
				\item[QA60] Jeder Benutzer kann in bis zu 20 Gruppen Mitglied sein
				\item[QA70] Der Server unterstützt das Anlegen von mindestens 30000 Benutzern
			\end{itemize}
	
	
	\newpage
	\section{Verzögerungen}
	1. geplantes Diagramm
	2. "richtiges" Diagramm
	
	\newpage
	\section{Unit-Tests}
	Hier Übersicht über Unit-Tests bzw. andere Tests der Phase
	\subsection{Server}
	\subsubsection{Datenbank}
	\subsubsection{Algorithmus}
	\subsubsection{Servlet}
	
	\subsection{Client}
	\subsubsection{HTTPConnection}
	\subsubsection{Model}
	\subsubsection{View}
	\subsubsection{Controller}
\end{document}
