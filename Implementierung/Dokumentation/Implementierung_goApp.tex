\documentclass{scrartcl}
\usepackage[utf8]{inputenc}
\usepackage[T1]{fontenc}
\usepackage{verbatim}
\usepackage{enumitem}
\usepackage{graphicx}
\usepackage[ngerman]{babel}
\usepackage{hyperref}
\usepackage{xcolor}
\usepackage{pdfpages}
\setlength{\parindent}{0em} 
\hypersetup{
    colorlinks,
    linkcolor={blue!65!black},
    citecolor={blue!50!black},
    urlcolor={blue!80!black}
}
\title{goApp Implementierung}
\author{Jörn Kussmaul, Katharina Riesterer, Julian Neubert,\\ Jonas Walter, Tobias Ohlsson, Eva-Maria Neumann}
\begin{document}
	\maketitle
	\newpage
	\tableofcontents
	\newpage

	\section{Einleitung}
	Anschluss auf Plichtenheft \& Entwurf
	
	\newpage
	\section{Änderungen am Entwurf}
	Hier jeder (sofort) Eintragen, wenn er was anders macht, wie eigentlich geplant.
	
	\subsection{ServletUtils}
	-Methoden werden öfters in verschiedenen Servlet genutzt
	-verringern von redundantem Code
	
	\subsection{JSONParameter}
	-unter Enums hinzugefügt für Fehler und Methoden
	\newpage
	\section{Kriterien}
	\subsection{Musskriterien}
	\subsection{Wunschkritereien}
	
	\newpage
	\section{Verzögerungen}
	1. geplante Diagramm
	2. "richtige" Diagramm
	
	\newpage
	\section{Unit-Tests}
	Hier Übersicht über Unit-Tests bzw. andere Tests der Phase
	\subsection{Server}
	\subsection{Client}
	
\end{document}
