\documentclass{scrartcl}
\usepackage[utf8]{inputenc}
\usepackage[T1]{fontenc}
\usepackage{verbatim}
\usepackage{enumitem}
\usepackage{graphicx}
\usepackage[ngerman]{babel}
\usepackage{hyperref}
\usepackage{xcolor}
\usepackage{pdfpages}
\setlength{\parindent}{0em} 
\setcounter{secnumdepth}{4}
\hypersetup{
    colorlinks,
    linkcolor={blue!65!black},
    citecolor={blue!50!black},
    urlcolor={blue!80!black}
}
\title{goApp Qualitätssicherung}
\author{Jörn Kussmaul, Katharina Riesterer, Julian Neubert,\\ Jonas Walter, Tobias Ohlsson, Eva-Maria Neumann}
\begin{document}
	\maketitle
	\newpage
	\tableofcontents
	\newpage

\section{Einleitung}
\newpage
\section{Server}
\newpage
\section{Client}
	\subsection{Lint}
	Lint ist ein Programm zur statischen Code-Analyse der Quelltexte von Computerprogrammen und in Android Studio integriert. Lint überprüft Android-Projekten auf mögliche Bugs und liefert Vorschläge um den Code zu verbessern.
Wir haben unseren Client Code mehrmals mit Lint überprüft und dadurch immer weiter optimiert, von Performance Issues über Rechtschreibfehler bis hin zur Verbesserung der XML-Dateien.  

	\subsection{Performance}
		\subsubsection{Monkey-Test}
		\subsubsection{Beta Tester}
		Wir haben unsere App unter circa xx Beta Usern verteilt um sie unter realen Bedingungen durch Personen testen zu lassen welche nicht an der Entwicklung beteiligt waren und um Verbesserungsvorschläge an der GUI zu sammeln und umzusetzten.
		\newline Einige Beispiele:
			\begin{itemize}
				\item User bekommen jetzt bei "`kritischen"' Aktionen wie dem Löschen einer Gruppe einen Dialog angezeigt ob sie diese Aktion wirklich durchführen wollen.
				\item Alle Texte der goApp sind jetzt eingerückt und kleben nicht mehr am Bildschirmrand.
				\item Die Reihenfolge der Buttons in der Toolbar der StartActivity wurde geändert da sie so intuitiver zu bedienen sind.
			\end{itemize}
		
		Ebenfalls waren wir an eventuellem Fehlverhalten oder gar Abstürzen der Applikation im "`alltagsgebrauch"' interessiert und haben auch hier Feedback gesammelt um Bugs zu beheben.

		\begin{itemize}
			\item Problem: Teilweise stürzte die goApp ab wenn der User eine Gruppe löscht.
			\begin{itemize}
				\item Fehler: Nachdem ein User eine Gruppe löscht soll er auf die StartActivity gelangen, weil seine Gruppe nun nicht mehr existiert. Um eine Activity zu starten muss ein Intent mit dem Context erstellt werden durch den die  Activity gestartet werden soll. Die StartActivity wird hier aus dem ResultReceiver eines Fragments gestartet, hier nutzten wir die Methode getActivity() um einen Kontext zu bekommen. Diese Mehtode lieferte meist auch die GroupInfoActivity, manchmal wurde aber auch ein null Objekt zurückgegeben.
				\item Lösung: Der ResultReceiver erhält jetzt die GroupInfoActivity als Parameter für den Konstruktor und speichert sich diese, so kann er mit diesem Kontext immer eine andere Activity starten.
			\end{itemize}

			\item Problem: Teilweise stürzte die goApp ab wenn der User eine Gruppe verlässt. Fehler und Lösung sind wie beim Löschen einer Gruppe.

			\item Problem: Wenn ein User erfolgreich eine Gruppe gelöscht hat konnte mit dem Zurückbutton wieder in die GroupInfoActivity der entsprechenden Gruppe zurückgekehrt werden.
			\begin{itemize}
				\item Fehler: Die GroupInfoActivity war noch auf dem back stack der zuletzt geöffneten Activities abgelegt. 
				\item Lösung: Wenn die StartActivity gestart wird wird der komplette back stack gelöscht.
			\end{itemize}

			
		\end{itemize}
	
		Im Anhang findet sich zusätzlich ein Fragebogen an die Beta User.




\end{document}